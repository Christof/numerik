\section{Numerische Integration}
Ziel: Näherungsweise Berechnung von bestimmten Integralen
durch ``Quadraturformeln'', z.B. durch:
\begin{align*}
  \int_a^b f(x) \dx \simeq I_{[a,b]]}^{(n)}(f) := \sumizn{w_i f(x_i)}
\end{align*}
wobei $w_i$ Integrationsgewichte und $x_i$ Stützstellen sind.
\para{Definition:} Der Genauigkeitsgrad einer QF $I_{[a,b]}^{(n)}$ ist
die größte Zahl $r \in \mathbb{N}_0$, für die gilt:
\begin{align*}
  \int_a^b f(x) \dx = I_{[a,b]}^{(n)}(p)\,\forall\,p \in \Pi_r
\end{align*}

\subsection{Interpolatorische Quadraturformeln 1 (IQF)}
Konstruktion der QF über Interpolation von $f$ in $a \leq x_0 < \ldots < x_n \leq b$.
\begin{align*}
  \int_a^b f(x) \dx &\approx \int_a^b \underbrace{p_r(x)}_{\text{IP}} \dx = 
  \int_a^b \sumizn{f(x_i)L_i(x)\dx} = \sumizn{f(x_i) \int_a^b L_i(x)\dx} \tag{4.1}\\
  &= \sumizn{f(x_i) \int_a^b \prod_{j=0,i\neq j}^n \frac{x-x_j}{x_i - x_j}\dx}
\end{align*}
$w_i$ sind nicht abhängig von $f$. Fehler von (4.1.):
\begin{align*}
  \abs{\int_a^b f(x)\dx - \int_a^b p_n(x)\dx} &= \abs{ \int_a^b \frac{1}{(n+1)!} f^{(n+1)}(\xi(x))(x-x_0)\cdots(x-x_n)\dx}\\
  &\leq \frac{1}{(n+1)!} \norm{f^{(n+1)}}_\infty \underbrace{\norm{(x-x_0)\cdots(x-x_n)}_\infty}_{(b-a)^{n+1}} \underbrace{\int_a^b 1 \dx}_{b-a}\\
  &\leq \frac{\norm{f^{(n+1)}}_\infty}{(n+1)!}(b-a)^{n+2}
\end{align*}
Genauigkeitsgrad einer IQF $I_{[a,b]}^{(n)}$: mindestens $n$

\subsubsection{Geschlossene Newton-Cotes-Formeln}
Sind IQF mit gleichmäßig verteilten Stützstellen $x_i = a + ih$ $i=0,\ldots,n$ $h=\frac{b-a}{n}$.
(``Geschlossen'', weil $a$ und $b$ als Stützstellen verwendet werden.)\\
Beispiel: $n=1$ $x_0=a$ $x_1=b$
\missingfigure{IQF n=1}
\begin{align*}
  \int_a^b f(x) \dx \approx \frac{1}{2} (b-a)[f(a)+f(b)] \tag{Trapezregel}\\
  w_0 = w_1 = \frac{1}{2} (b-a)
\end{align*}
Integrationsgewichte ($n$ allgemein)
\begin{align*}
  w_i &= \int_a^b L_i(x) \dx = \int_a^b \prod_{j=0,i\neq j}^n \frac{x-x_j}{x_i - x_j}\dx =\\
  &= h \int_{t(a) = 0}^{t(b) = n} \frac{a + th - (a+jh)}{a + ih - (a+jh)} \dvar{t} = h \int_0^t \prod \frac{(t-j)h}{i-j)h} \dvar{t}
\end{align*}
\begin{align*}
  t:= \frac{x-a}{h} && \frac{\dvar{t}}{\dx} = \frac{1}{h} \\
  t(a)=0 && t(b)=\frac{b-a}{h}=n && x_j = n + jh
\end{align*}

\subsection{Offene Newton-Cotes-Formeln}
Sind IQF mit $x_i = a + (i+1)h$ $i=0,\ldots,n \in \mathbb{N}_0$
$h=\frac{b-a}{n+2}$
\missingfigure{offene newton-cotes-formeln n=0, n=1}
Beispiel: $n=0$
\missingfigure{n=1 beispiel}
\begin{align*}
  \int_a^b f(x) \dx \approx \underbrace{(b-a)}_{w_0} f(\frac{a+b}{2}) \tag{Mittelpunkt-/Rechteckregel}
\end{align*}
Warnung: Bei Newton-Cotes-Formeln treten bei $n \geq 8$ (geschlossene) und $n \geq 2$ (offene)
negative Gewichte auf, was zur Auslöschung führen kann. (Oft: $f \geq 0$ oder $f \leq 0$).
Daher für solche $n$ nie verwenden.

\subsection{Summierte Quadraturformeln}
Idee: Unterteile $[a,b]$ in $N$ Teilintervalle. Wende auf jedem Teilintervall vorgebene QF an.
Sinnvoll (notwendig): Fehler hängt von $(b-a)$ ab.
\para{Summierte Trapezregel}
Unterteile $[a,b]$ in $N$ Teilintervalle der Länge $h=\frac{b-a}{N}$ $x_i := a + ih$ $i=0,\ldots,N$.
\begin{align*}
  \int_a^b f(x) \dx &= \sum^{N}_{i=1} \int_{x_{i-1}}^{x_i} f(x) \dx\\
  &\approx \sum^{N}_{i=1} \frac{h}{2} \left[f(x_{i-1}) + f(x_i)\right] \\
  &= h\left[\frac{1}{2} f(a) + \sum^{N-1}_{i=1} f(x_i) + \frac{1}{2} f(b)\right] =: I_h^{(1)}(f)
\end{align*}
\para{Summierte Simpsonregel}
\missingfigure{summierte simpsonregel}
\begin{align*}
  \int_a^b f(x) \dx &= \sum^{N}_{i=1} \int_{x_{i-1}}^{x_i} f(x) \dx\\
  &\approx \sum^{N}_{i=1} \frac{h}{6} \left[f(x_{i-1}) + 4f(\frac{x_{i-1}+x_i}{2}) + f(x_i)\right] \\
  &= \frac{h}{6} \left[ f(a) + 2 \sum^{N-1}_{i=1} f(x_i) + 4 \sum^{N}_{i=1} \frac{x_{i-1}+x_i}{2} + f(b) \right] := I_h^{(2)}(f)
\end{align*}
\para{Satz:}
\begin{enumerate}[a)]
  \item Für $f \in C^2[a, b]$ gilt: \begin{align*}
    \abs{\int_a^b f(x) \dx - I_h^{(1)}(f)} \leq ch^2 (b-a) \norm{f''}_\infty \underset{n \to \infty}{\longrightarrow} 0
  \end{align*}
  \item Für $f \in C^4[a, b]$ gilt: \begin{align*}
      \abs{\int_a^b f(x) \dx - I_h^{(2)}(f)} \leq ch^4 (b-a) \norm{f^{(4)}}_\infty \underset{n \to \infty}{\longrightarrow} 0
  \end{align*}
\end{enumerate}
Beweis für a):
\begin{align*}
  \abs{\int_a^b f(x) \dx - I_h^{(1)}(f)} \leq c \sum^{N}_{i=1} h^3 \norm{f''}_\infty = c \underbrace{h^3}_{\frac{(b-a)^3}{N^3}} N \norm{f''}_\infty = c h^2 (b-a) \norm{f''}_\infty
\end{align*}
