\subsection{Übung 1}
\subsubsection{Beispiel 1}
$ \Fhat := (10,6,-9,9) $ \\
Kerngrößen:\\
$ x_{min}(\F) = 0.1 \cdot 10^{-9} = 10^{-10} $ \\
$ x_{min}(\Fhat) = 0.000001 \cdot 10^{-9} = 10^{-15} $ \\
$ x_{max}(\F) = 0.999999 \cdot 10^{9} $ \\
$ eps = \frac{1}{2} b^{-t + 1} = \frac{1}{2} 10^{-5} $ \\
$ \epsilon_{M} = 2eps = 10^{-5} $ \\
\\
$ \pi = 3.14159265\cdots $ \\
$ rd(\pi) = \tilde{\pi} = 0.314159 \cdot 10^1 $ \\
relativer Fehler: $ | \frac{\tilde{\pi} - \pi}{\pi} | = 8.45 \cdot 10^{-7} $

\subsubsection{Beispiel 2}
$ \Fhat := (10,4,-9,9) $ \\
$ eps = \frac{1}{2} 10^{-3} $ \\
$ c_1 = \sqrt{2(1-\cos(\phi))} $ \\
$ c_2 = 2 \sin(\frac{\phi}{2}) $ \\
$ c_2 = 0.023270531\cdots $ \\
$ \phi= \frac{4}{3} \;TODOabstand \tilde{\phi} = 0.1333 \cdot 10 = 1.333 $ \\

$ \tilde{c_1}: $ \\
$ rd({\cos}(\tilde{\phi})) = 0.9997 $ \\
$ rd(1 - 0.9997) = 0.0003 $ \\
$ rd(2 \cdot 0.0003) = 0.0006 $ \\
$ rd(\sqrt{0.0006}) = 0.02449 = \tilde{c_1} $ \\
relativer Fehler: $ | \frac{\tilde{c_1} - c_2}{c_2} | = 5.24\% $ \\

$ \tilde{c_2}: $ \\
$ rd(\frac{\tilde{\phi}}{2}) = 0.6665 $ \\
$ rd(\sin(0.6665)) = 0.01163 $ \\
$ rd(2 \cdot 0.01163) = 0.02326 = \tilde{c_2} $ \\
relativer Fehler: $ | \frac{\tilde{c_2} - c_2}{c_2} | = 0.04525\% $ \\

