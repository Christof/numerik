\section{Numerische Lineare Algebra}
\textbf{ZIEL}: Beantwortung folgender Fragen:
\begin{itemize}
\item{Gibt es Algorithmen, so dass grosse Abweichungen der berechneten Loesung (Lsg) zur tatsaechlichen nicht auftreten?}
\item{Kann man vorhersagen wann es zu grossen Fehlern kommt, und wann nicht?}
\end{itemize}

\begin{equation*}
  \left.
    \begin{align*}
      \exists \widetilde{x} \neq 0: A\widetilde{x} = 0 & \Rightarrow  Ax = b \\
        & \Rightarrow A(x + \alpha \widetilde{x}) = b
%      \exists \widetilde{x} \neq 0: A\widetilde{x} = 0 \Rightarrow & Ax = b  \\
%      & A(x + \alpha \widetilde{x}) = b
%		\Fhat = \F \cup \F_d \hspace{0.5cm} \text{mit} \hspace{0.5cm} & \F && \text{normalisierte Gleitpunktzahlen}\\
%		 &\F_d && \text{denormalisierte Gleitpunktzahlen} 
    \end{align*}
  \right\}
  \text{linear aber nicht eindeutig}
\end{equation*}

\subsection{Grundlegendes}

\subsubsection{Matrix Norm}

Jede Norm über $\mathbb{K}^{m \times n}$ heißt Matrixnorm.
Matrixnormen lassen sich durch Vektornormen induzieren:
\begin{equation*}
\|A\|_p \coloneqq \underset{x \neq 0}{max} \frac{\|Ax\|_p}{\|x\|_p} = \underset{\|x\|_p = 1}{max}\|Ax\|_p
\end{equation*}

z.B.: $p \in \left\{1, 2, \inf \right\}$

Für induzierte Normen gilt:

\begin{itemize}
\item{$\|Ax\|_p \leq \|A\|_p \|x\|_p$}
\item{$\|A B\|_p \leq \|A\|_p \|B\|_p$}
\end{itemize}
Lässt sich einfach beweisen.

Im folgenden werden nur induzierte Normen verwendet

Satz: Sei $A \in \mathbb{K}^{m \times n}$. Dann gilt:
\begin{enumerate}
\item{$\|A\|_1 = \underset{j = 1,..,n}{max} \sum\limits_{i=1}^{m}{\left|a_{ij}\right|}$ } %Spalten Summennorm
\item{$\|A\|_{\infty} = \underset{j = 1,..,m}{max} \sum\limits_{i=1}^{n}{\left|a_{ij}\right|}$ } %Zeilen SummennormS
\item{$\|A\|_2 = \sqrt{\lambda_{max}\left(A^*A\right)}$}
\end{enumerate}
Beweise a, b durch einsetzen, c benötigt Mittel aus linearen Algebra.

\subsubsection{$\left(Relative\right)$ Kondition einer Matrix}
%\begin{align*}
%Geg.: &Sei $Ax = b$ (ungestörtes System), wobei \\
      %&$A \in \mathbb{K}^{m \ times n}$ invertierbar und $b \in \mathbb{K}^n$ sei.  \\
			%&Wie wirken sich Störungen $\Delta A$ und $\Delta b$ aus:  \\
			%&$\left(A+\DeltaA)(x+\Delta x)=b+\Delta b$  Gestörte System
%\end{align*}
...

Berechnung von $\kappa_2\left(A\right)$
\begin{equation*}
  \begin{align*}
    \kappa_2(A) = \|A\|_2 \|A^{-1}\|_2 = &\sqrt{\lambda_{max}\left(A^*A\right)} \cdot \frac{1}{\sqrt{\lambda_{min}\left(A^*A\right)}} //
    & \frac{\sigma_1\left(A\right)}{\sigma_n\left(A\right)}
  \end{align*}
\end{equation*}
Für symmetrische und positiv definite $\frac{\sigma_1\left(A\right)}{\sigma_n\left(A\right)}$ Matrizen:
\begin{equation*}
  \kappa_2(A) = \frac{\lambda_{max}\left(A\right)}{\lambda_{min}\left(A\right)}
\end{equation*}

Bemerkung: Man beachte, dass in diesem Abschnitt die Konditionszahl einer Matrix $A$ definiert wurde und nicht die des Problems:

$Input\left(A,B\right) \rightarrow Output\left(A^{-1}b\right)$

Letzteres kann über den Ansatz in Abschnitt  berechnet werden und hängt eng mit $\kappa_A$ zusammen.

\textbf{Zeilenskalierung}
Bsp.: NUR DEMO!
\begin{equation*}
  \begin{align*}
    \hspace{1cm} &Ax = \begin{pmatrix} 10^8 & 0 \\ 0 & 10^{-8} \end{pmatrix} x = \begin{pmatrix}b_1 \\ b_2\end{pmatrix} \\
    &\kappa_\infty\left(A\right) = 10^8 \cdot 10^8 = 10^{16}
  \end{align*}
\end{equation*}
Vorkonditionierung durch Diagonalmatrix
\begin{equation*}
  D = \begin{pmatrix}10^{-8} & 0 \\ 0 & 10^8\end{pmatrix},
\end{equation*}
d.h. Löse statt Ax = b, dass dazu äquivalente System
\begin{equation*}
  \begin{align*}
    &DAx = Db \\
    &\kappa_\infty\left(DA\right) = \kappa_\infty\left(I\right) = 1
	\end{align*}	
\end{equation*}

Verbesserung der Konditionszahl bezgl. $\|.\|_\infty$ durch Zeilenskalierung.
Definiere: Diagonalmatrix:
\begin{equation*}
  D_z \left[ i,i \right] \coloneqq \left(\sum\limits_{j=1}^{n}{\left|a_{ij}\right|}\right)^{-1} %
\end{equation*}
Dann gilt:
\begin{equation*}
  \begin{align*}
    &\sum\limits_{j=1}^{n}{\left|D_zA\left[i,j\right]\right|} = 1  \hspace{2cm} fuer 1, ..., n \\
    &\Rightarrow \|D_zA\|_\infty = 1
		\end{align*}
\end{equation*}