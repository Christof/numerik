\section{Numerische Lineare Algebra}
\textbf{ZIEL}: Beantwortung folgender Fragen:
\begin{itemize}
\item{Gibt es Algorithmen, so dass grosse Abweichungen der berechneten Loesung (Lsg) zur tatsaechlichen nicht auftreten?}
\item{Kann man vorhersagen wann es zu grossen Fehlern kommt, und wann nicht?}
\end{itemize}

\begin{equation*}
  \left.
    \begin{aligned}
      \exists \widetilde{x} \neq 0: A\widetilde{x} = 0 & \Rightarrow  Ax = b \\
        & \Rightarrow A(x + \alpha \widetilde{x}) = b
%      \exists \widetilde{x} \neq 0: A\widetilde{x} = 0 \Rightarrow & Ax = b  \\
%      & A(x + \alpha \widetilde{x}) = b
%		\Fhat = \F \cup \F_d \hspace{0.5cm} \text{mit} \hspace{0.5cm} & \F && \text{normalisierte Gleitpunktzahlen}\\
%		 &\F_d && \text{denormalisierte Gleitpunktzahlen} 
    \end{aligned}
  \right\}
  \text{linear aber nicht eindeutig}
\end{equation*}

\subsection{Grundlegendes}

\subsubsection{Matrix Norm}

Jede Norm über $\mathbb{K}^{m \times n}$ heißt Matrixnorm.
Matrixnormen lassen sich durch Vektornormen induzieren:
\begin{equation*}
\|A\|_p \coloneqq \underset{x \neq 0}{max} \frac{\|Ax\|_p}{\|x\|_p} = \underset{\|x\|_p = 1}{max}\|Ax\|_p
\end{equation*}

z.B.: $p \in \left\{1, 2, \inf \right\}$

Für induzierte Normen gilt:

\begin{itemize}
\item{$\|Ax\|_p \leq \|A\|_p \|x\|_p$}
\item{$\|A B\|_p \leq \|A\|_p \|B\|_p$}
\end{itemize}
Lässt sich einfach beweisen.

Im folgenden werden nur induzierte Normen verwendet

\textbf{Satz:}
Sei $A \in \mathbb{K}^{m \times n}$. Dann gilt:
\begin{enumerate}
\item{$\|A\|_1 = \underset{j = 1,..,n}{max} \sum\limits_{i=1}^{m}{\left|a_{ij}\right|}$ } %Spalten Summennorm
\item{$\|A\|_{\infty} = \underset{j = 1,..,m}{max} \sum\limits_{i=1}^{n}{\left|a_{ij}\right|}$ } %Zeilen SummennormS
\item{$\|A\|_2 = \sqrt{\lambda_{max}\left(A^*A\right)}$}
\end{enumerate}
\textbf{Beweise:} a, b durch einsetzen, c benötigt Mittel aus linearen Algebra.

\subsubsection{$\left(Relative\right)$ Kondition einer Matrix}
%\begin{aligned}
%Geg.: &Sei $Ax = b$ (ungestörtes System), wobei \\
      %&$A \in \mathbb{K}^{m \ times n}$ invertierbar und $b \in \mathbb{K}^n$ sei.  \\
			%&Wie wirken sich Störungen $\Delta A$ und $\Delta b$ aus:  \\
			%&$\left(A+\DeltaA)(x+\Delta x)=b+\Delta b$  Gestörte System
%\end{aligned}
...

Berechnung von $\kappa_2\left(A\right)$
\begin{equation*}
  \begin{aligned}
    \kappa_2(A) = \|A\|_2 \|A^{-1}\|_2 = &\sqrt{\lambda_{max}\left(A^*A\right)} \cdot \frac{1}{\sqrt{\lambda_{min}\left(A^*A\right)}} //
    & \frac{\sigma_1\left(A\right)}{\sigma_n\left(A\right)}
  \end{aligned}
\end{equation*}
Für symmetrische und positiv definite $\frac{\sigma_1\left(A\right)}{\sigma_n\left(A\right)}$ Matrizen:
\begin{equation*}
  \kappa_2(A) = \frac{\lambda_{max}\left(A\right)}{\lambda_{min}\left(A\right)}
\end{equation*}

Bemerkung: Man beachte, dass in diesem Abschnitt die Konditionszahl einer Matrix $A$ definiert wurde und nicht die des Problems:

$Input\left(A,B\right) \rightarrow Output\left(A^{-1}b\right)$

Letzteres kann über den Ansatz in Abschnitt  berechnet werden und hängt eng mit $\kappa_A$ zusammen.

\textbf{Zeilenskalierung}
Bsp.: NUR DEMO!
\begin{equation*}
  \begin{aligned}
    \hspace{1cm} &Ax = \begin{pmatrix} 10^8 & 0 \\ 0 & 10^{-8} \end{pmatrix} x = \begin{pmatrix}b_1 \\ b_2\end{pmatrix} \\
    &\kappa_\infty\left(A\right) = 10^8 \cdot 10^8 = 10^{16}
  \end{aligned}
\end{equation*}
Vorkonditionierung durch Diagonalmatrix
\begin{equation*}
  D = \begin{pmatrix}10^{-8} & 0 \\ 0 & 10^8\end{pmatrix},
\end{equation*}
d.h. Löse statt Ax = b, dass dazu äquivalente System
\begin{equation*}
  \begin{aligned}
    &DAx = Db \\
    &\kappa_\infty\left(DA\right) = \kappa_\infty\left(I\right) = 1
	\end{aligned}	
\end{equation*}

Verbesserung der Konditionszahl bezgl. $\|.\|_\infty$ durch Zeilenskalierung.
Definiere: Diagonalmatrix:
\begin{equation*}
  D_z \left[ i,i \right] \coloneqq \left(\sum\limits_{j=1}^{n}{\left|a_{ij}\right|}\right)^{-1} %
\end{equation*}
Dann gilt:
\begin{equation*}
  \begin{aligned}
    &\sum\limits_{j=1}^{n}{\left|D_zA\left[i,j\right]\right|} = 1  \hspace{2cm} fuer 1, ..., n \\
    &\Rightarrow \|D_zA\|_\infty = 1
		\end{aligned}
\end{equation*}

\textbf{Satz:}
$\kappa_\infty\left(D_zA\right) \leq \kappa_\infty\left(DA\right)$ \hspace{2cm} $\forall$ regulären Diagonalmatrizen

\textbf{Beweis:}
Sei $A$ bereits äquilibriert, d.h.:
\begin{equation*}
  \sum\limits_{j=1}^{n}{\left|A\left[i,j\right]\right|} = 1 \hspace{2cm} fuer i = 1, .., n \\	\\
	\Rightarrow \|A\|_\infty = 1 \\ \\
	\Rightarrow \kappa_\infty\left(A\right) = \|A\|_\infty \|A^{-1}\|_\infty = \|A^{-1}\|_\infty
\end{equation*}

Sei $D$ eine beliebige reguläre Diagonalmatrix
\begin{equation*}
\begin{aligned}
  \|DA\|_\infty = &\underset{1 \leq i \leq n}{max} \left\{\sum\limits_{j=1}^{n}{\left|d_ia_{ij}\right|}\right\} =
  \underset{1 \leq i \leq n}{max}\left\{\left|d_i\right|\sum\limits_{j=1}^{n}{\left|a_{ij}\right|}\right\} = \\
  &\underset{1 \leq i \leq n}{max}\left\{\left|d_i\right|\right\} = \|D\|_\infty = 
  \|\left(DA\right)^{-1}\|_\infty = \underset{x \neq 0}{max}\left\{\frac{\|A^{-1}D^{-1}x\|_\infty}{\|x\|_\infty}\right\} = \\
  &\underset{y \neq 0}{max}\left\{\frac{\|A^{-1}y\|_\infty}{\|Dy\|_\infty}\right\} \geq \underset{y \neq 0}{max}\left\{\frac{\|A^{-1}y\|_\infty}{\|D\|_\infty\|y\|_\infty}   \right\} \\
  &\Rightarrow \kappa_\infty\left(DA\right) = \|DA\|_\infty \|\left(DA\right)^{-1}\|_\infty \geq \\
  &\hcancel{$\|D\|_\infty$} \|A^{-1}\|_\infty \hcancel{$\|D\|_{\infty}^{-1}$} = \kappa_\infty\left(A\right)
\end{aligned}
\end{equation*}

\subsubsection{Residuum}
Sei $\widetilde
{x}$ eine Näherungslösung von GLS
\begin{equation*}
Ax=b
\end{equation*}

Dann bezeichnet man
\begin{equation*}
r \coloneqq r_{\widetilde{x}} \coloneqq A\widetilde{x} - b
\end{equation*}

als Residuum.
\textbf{ACHTUNG:}
\begin{itemize}
  \item{Residuum ist nicht der Fehler von $\widetilde{x}$}
	\item{Für die exakte Lösung $x$ gilt:}
\end{itemize}
\begin{equation*}
r_x = Ax - b = 0
\end{equation*}

Frage: Ist $r_{\widetilde{x}}$ ein guter Indikator dafür, wie genau $\widetilde{x}$ ist?
Antw.: Nein, außer Matrix ist gut Knoditioniert!

\textbf{Satz:}
Sei $a \in \mathbb{K}^{n \times n}$ invertierbar und $\widetilde{x}$ Näherungslösung von
\begin{equation*}
\begin{aligned}
Ax = b \hspace{2cm} b \neq 0
\end{aligned}
\end{equation*}
Dann gilt:
\begin{equation*}
\frac{\|x - \widetilde{x}\|}{\|x\|} \leq \kappa\left(A\right) \frac{\|r\|}{\|b\|}
\end{equation*}
\textbf{Beweis:}
\begin{equation*}
\begin{aligned}
r &= A\widetilde{x} - b = A\widetilde{x} - Ax = A\left(\widetilde{x} - x\right) \\
&\Rightarrow \widetilde{x} - x = A^{-1}r \\
&\|\widetilde{x} - x\| = \|A^{-1}r\| \leq \|A^{-1}\|\|r\| \\
&\|b\| = \|Ax\| \leq \|A\|\|x\| \\
&\frac{1}{\|x\|} \leq \frac{\|A\|}{\|b\|}
\end{aligned}
\end{equation*}

\subsection{Direkte Lösung von Linearen Gleichungssystemen (in n - Schritten exakte Lsg)}
\subsubsection{Gauß - Algorithmus und LR - Zerlegung}
Zunächst: &Voraussetzung das Gauß - Algorithmus ohne Zeilen verauschen druchführbar sei:

\begin{equation*}
  \begin{aligned}
    Ax = b \\
    &a_{11}^{\left(1\right)}x_1 + \ldots + a_{1n}^{\left(1\right)}x_n = b_1^{\left(1\right)}  \\
    &\vdots                               \vdots                     = \vdots  \\
    &a_{n1}^{\left(1\right)x_1 + \ldots + a_{nn}^{\left(1\right)}x_n = b_n^{\left(1\right)}  \\
  \end{aligned}
\end{equation*}

1.Schritt: Subtrahiere das $\left(\frac{a_{i1}}{a_{11}}\right) =: L_{i1}$ - fache der 1. Zeile
von der i-ten Zeile, i = 2, ..., n
\begin{equation*}
  \begin{aligned}
    &a_{11}^{\left(1\right)}x_1 + \ldots + a_{1n}^{\left(1\right)}x_n = b_1^{\left(1\right)}  \\
		&\vdots                    +a_{22}^{\left(2\right)}x_2 \ldots + a_{1n}^{\left(2\right)}x_n = b_2^{\left(2\right)}  \\
    &\vdots                               \vdots                     = \vdots  \\
    &\ldots  + a_{2n}^{\left(2\right)}x_2 + a_{nn}^{\left(2\right)}x_n = b_n^{\left(2\right)} 
		%
		%
		A^{\left(2\right)} = L-1A^{\left(1\right)}, wobei dieses bleibt gleich
    L_1 = \begin{pmatrix} 1 & \ldots &        &  \\ 
		                -L_{21} & 1      & 0      &  \\
					          \vdots	& 0      & \ddots &  \\
										-L_{n1} & \ldots &        & 1
					\end{pmatrix}
					
		s. Schritt: A^{\left(s\right)} = L_{s-1}A^{\left(s-1\right)} \\
		A_s = \begin{pmatrix}
		      a_{11}^{\left(1\right)} & a_{12}^{\left(1\right)} & \ldots & \ldots & a_{1n}^{\left(1\right)} \\
					\vdots                  & a_{22}^{\left(2\right)} & \ldots & \ldots & a_{2n}^{\left(2\right)} \\
					\vdots                  & \ddots                  & \ldots & \ldtos & \vdots                  \\
					                        &                         & a_{ss}^{\left(s\right)} & \ldots  & a_{sn}^{\left(s\right)} \\
																	&                         & a_{ns}^{\left(s\right)} & \ldots  & a_{nn}^{\left(s\right)}                                 
		      \end{pmatrix}
					b = \begin{pmatrix}b_1^{\left(1\right)} \\ b_2^{\left(2\right)} \\ \vdots \\ b_s^{\left(s\right)} \\ b_n^{\left(s\right)}\end{pmatrix}
					
		L_s = \begin{pmatrix}
		      1 &   &        &            &        &        & \\
					  & 1 &        &            &        &        & \\
						&   & \ddots &            &        &        & \\
						&   &        & 1          &        &        & \\
						&   &        & -L_{s+1,s} & \ddots &        & \\
						&   &        & \vdots     &        & \ddots & \\
						&   &        & -L_{ns}    &        &        & 1
		      \end{pmatrix}
					
		L_{i,s} = \frac{a_{i,s}^{\left(s\right)}}{a_{s,s}^{\left(s\right)}    fuer i = s + 1, ..., n
  \end{aligned}
\end{equation*}

Schritt n-1: $A^{\left(n\right)} = L_{n-1}A^{n-1}$
Rechte obere

