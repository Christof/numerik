%\section{Approximation von Funktionen}
% hat er es nur umbenannt?, ja glaub schon
\section{Interpolation}
Aufgabenstellung: Aus einer festgelegten Menge von Funktionen $M_n$ 
bestimme man eine Funktion, die durch die gegebenen Punkte
$(x_0, f_0), (x_1, f_1), \cdots, (x_n, f_n) \in \mathbb{R}^2$ verläuft.

\missingfigure{Funktion mit Stützstellen}
Die Wahl von $M_n$ ist abhängig von der Problemstellung:
\begin{itemize}
  \item $\Pi_n$: Menge der Polynome mit Grad $\leq$ n
  \item stückweise polynomiale Funktion
  \item trigonometrische Funktion
	\item $\cdots$
\end{itemize}
Warum und weshalb:
\begin{itemize}
  \item Berechnung von Zwischenwerten einer Funktion, die nur an wenigen 
    Stellen bekannt ist
  \item Vereinfachung der Komplexität einer Funktion. (Beschreibung
    einer Funktion durch eine kleine Anzahl von Funktionen) $\Rightarrow$
    einfacheres Rechnen
  \item wichtige theoretische Grundlage für verschiedene andere numerische
    Aufgaben (Integration, Differenzialgleichungen)
\end{itemize}

\subsection{Polynominterpolation}
\underline{Gegeben}: Paarweise verschiedene Stützstellen $x_0, x_1, \cdots x_n$ und
Werte $f_0, f_1, \cdots f_n$.\\
\underline{Gesucht}:
\begin{equation*}
  \tag{2.1} p_n \in  \Pi_n \text{, so dass } p_n(x_i) = 
  f_i \text{ für } i = 0, 1, \cdots ,n
\end{equation*}
Grundlegende Fakten zu Polynomen:
\begin{enumerate}[(i)]
  \item $\Pi_n$ die Menge der Polynome mit Grad $\leq$ n ist ein Vektorraum
  \item Die Monome $1, x, x^2, \cdots, x^n$ bilden eine Basis von $\Pi_n$
  \item Polynom von Grad n $\geq$ 1 mit komplexen Koeffzienten besitzt genau n-Nullstellen
    in $\mathbb{C}$, wobei die Anzahl der Nullstellen entsprechend der Vielfachheit
    gezählt wird.
\end{enumerate}
\paragraph{Satz:} Die Polynominterpolationsaufgabe (2.1) ist eindeutig lösbar\\
Beweis:
\begin{enumerate}[(a)]
  \item Eindeutigkeit: Angenommen $p_n, q_n \in \Pi_n$ erfüllen (2.1), d.h. $p_n(x_i)=q_n(x_i)=f_i $für i=0,1,...n\\
    $r := p_n - q_n \in \Pi_n$ \\
    $r(x_i) = 0$ für $i = 0, \cdots, n \Rightarrow r$ hat $n + 1$ Nullstellen
    $\Rightarrow r \equiv 0 \Rightarrow p_n \equiv q_n$
  \item Existenz: Konstruiere Polynome $L_0(x), L_1(x), \cdots, L_n(x) \in \Pi_n$ mit\\
    $L_i(x_k)=\begin{cases} 1 & \mbox{für } \mbox{ $i = k$} \\ 
      0 & \mbox{sonst} \end{cases}$ \\
    $\Rightarrow L_i$ hat n Nullstellen: $x_0, x_1, \cdots, x_{i-1}, x_{i+1}, \cdots, x_n$
		\begin{align*}    
		L_i \in \Pi_n \Rightarrow L_i(x) = a(x-x_0)(x-x_1)\cdots(x-x_{i-1})(x-x_{i+1})\cdots(x-x_n)\\
    L_i(x_i) \overset{!}{=} 1 \Rightarrow
      a = \frac{1}{(x_i-x_0)(x_i-x_1)\cdots(x_i-x_{i-1})(x_i-x_{i+1})\cdots(x_i-x_n)}\\
    \end{align*}
		\begin{empheq}[innerbox=\fbox,right=\Leftarrow{\text{LAGRANGE-POLYNOME}}]{align*}
		\Rightarrow L_i(x) = \frac{(x - x_0)\cdots}{(x_i - x_0)\cdots} = 
      \prod\limits_{j = 0,\,j \neq i}^n \frac{x - x_j}{x_i - x_j} \\
		\end{empheq}
    \begin{equation*}
      \tag{2.2}
      p_n(x) = f_0 L_0(x) + f_1 L_1(x) + \cdots + f_n L_n(x) = 
      \sum\limits_{k = 0}^n f_k L_k(x)
    \end{equation*}
    $p_n(x_i) = 0 + 0 + \cdots + f_i\underbrace{L_i(x_i)}_{1} + 0 + \cdots = f_i$\\
\end{enumerate}
