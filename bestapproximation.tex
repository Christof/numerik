\section{Bestapproximation (BA)}
Aufgabenstellung:\\
Gegeben:
\begin{itemize}
  \item Element $f \in H$ (z.B. Funktion in $C[a,b]$)
  \item Raum $U$ für die Approximation (z.B. $\Pi_n$)
  \item Norm $\norm{.}$
\end{itemize}
Gesucht:
Bestapproximation $g \in U$ von $f$ bezüglich $\norm{.}$, so dass 
$\norm{f - g} \leq \norm{f - u}\,\forall\,u \in U$

\subsection{BA bezgl. einer durch ein Skalarprodukt induzierten Norm}
\para{Satz} Sei $H$ ein Vektorraum und $\inner{.}{.}$ ein Skalarprodukt auf $H$, das die
Norm $\norm{.}$ induziere. Weiters sei $U$ ein endlichdimensionaler Unterraum von $H$.
Dann gilt:
\begin{enumerate}[(a)]
  \item $g \in U$ ist BA von $f$ bzgl. 
    $\norm{.}\,\Longleftrightarrow\,\inner{f - g}{u} = 0\,\forall\,u \in U$
    (Orthogonalitätseigenschaft)
  \item $\forall f \in H$ gibt es genau eine BA $g \in U$ bzgl. $\norm{.}$
\end{enumerate}
Beweis: a) Einsetzen und Umformen
\missingfigure{Zeichnung mit Ebene und Projektion}
Beweis: b) (Wir führen den Bewis für $\mathbb{K} = \mathbb{R}$;
Analoger Beweis für $\mathbb{K} = \mathbb{C}$)\\
Konstruktionsbeweis: Sei ${\phi_0,\,\ldots,\,\phi_n}$ eine Basis von $U$ und $f \in H$\\
Ansatz: $g = \sumzn{j}{a_j \phi_j}$, wobei wir zeigen, dass $\ztoxn{a}$
eindeutig bestimmt sind, so dass $g$ BA von $f$ bzgl. $\norm{.}$ ist.
Teil a)
\begin{align*}
  g \in U \text{ ist BA von } f &\Longleftrightarrow \inner{f - g}{u} = 0\, \forall\, u \in U\\
  &\Longleftrightarrow\\
  \inner{f-g}{\phi_i} &= 0\, \forall\, i=0,\,\ldots,\,n\\
  &\Longleftrightarrow\\
  \inner{g}{\phi_i} &= \inner{f}{\phi_i}\, \forall\, i=0,\,\ldots,\,n\\
  \sumzn{j}{a_j\inner{phi_j}{\phi_i}} &= \inner{f}{\phi_i}\, \forall\, i=0,\,\ldots,\,n\\
  a_0\inner{\phi_0}{\phi_0} + a_1\inner{\phi_1}{\phi_0} + \ldots + a_n\inner{\phi_n}{\phi_0} &= \inner{f}{\phi_0}\\
    a_0\inner{\phi_0}{\phi_1} + a_1\inner{\phi_1}{\phi_1} + \ldots + a_n\inner{\phi_n}{\phi_1} &= \inner{f}{\phi_1}\\
  \ldots\\
  a_0\inner{\phi_0}{\phi_n} + a_1\inner{\phi_1}{\phi_n} + \ldots + a_n\inner{\phi_n}{\phi_n} &= \inner{f}{\phi_n}\\
  &\Longleftrightarrow\\
  M\underline{a} &= \underline{f}
\end{align*}
\begin{align*}
  M[i,j] &= \inner{\phi_j}{ \phi_i}\; i,j = 0,\,1,\,\ldots,\,n\\
    \underline{f}[i] &= \inner{f}{ \phi_i}\; i = 0,\,1,\,\ldots,\,n\\
  \underline{a}[i] &= a_i\; i = 0,\,1,\,\ldots,\,n\\
\end{align*}
M nennt man Gramsche Matrix zu $\ztoxn{\phi}$
