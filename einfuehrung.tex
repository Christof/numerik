\section{Einführung - 3.10.2012}

Numerische Mathematik befasst sich mit der näherungsweisen Lösung von Problemstellungen (mit dem Computer).
%\begin{itemize}
  %\item Formulierung
  %\item Entwicklung
  %\item Untersuchung
  %\item Implementierung
    %von Verfahren und Algorithmen
%\end{itemize}
\begin{equation*} 
  \left.
  \begin{aligned} 
   & \text{Formulierung} \\ 
   & \text{Entwicklung} \\ 
	 & \text{Untersuchung} \\
 	 & \text{Implementierung} \\
  \end{aligned} 
  \right\} 
  \text{von Verfahren und Algorithmen} 
\end{equation*} 
Von der Problemstellung zur Näherungslösung:

  \begin{empheq}{alignat*=3} 
   & \text{physikalisches, chemisches, ... Problem bzw. Modell} & \smarttag{Modellfehler} \\ 
   & \text{mathematisches Modell (z.B. DGL)} & \smarttag{Diskretisierungsfehler} \\ 
	 & \text{numerisches Näherungsverfahren (endlich} & \Smarttag{Verfahrensfehler}\\
	 & \text{ dim. Ersatzproblem)} \\
 	 & \text{Algorithmen für Ersatzprobleme} & \smarttag{Impl.-fehler z.B. Rundungsfehler}\\
	 & \text{Implementierungsfehler} & \\
  \end{empheq} 
\textbf{ZIEL}: (möglichst) genau, möglichst schnell, mit möglichst geringem Aufwand (z.B. Speicher)

\subsection{Gleitpunktzahlen $\Fhat$}
\begin{align*}
		\Fhat = \F \cup \F_d \hspace{0.5cm} \text{mit} \hspace{0.5cm} & \F && \text{normalisierte Gleitpunktzahlen}\\
		 &\F_d && \text{denormalisierte Gleitpunktzahlen} 
\end{align*}
\para{Normalisierte GPZ}

\begin{align*}
x = \sigma\,\cdot\,M\,\cdot\,\,b^e \hspace{0.5cm} \text{mit} \hspace{0.5cm} & \sigma && \text{Vorzeichen}\\
& M && \text{Mantisse} \\
& b && \text{Basis} \\
& e && \text{Exponent}
\end{align*}

\begin{align*}
M = \sum^t_{k=1} d_k b^{-k} \hspace{0.5cm} \text{mit} \hspace{0.5cm} & d_1,\cdots,d_t\, \in\,\{0,1,\cdots,b-1\} && \text{Ziffern der Mantisse}\\
& t && \text{Mantissenlänge} \\
& && d_1 \neq 0 \text{(Normalisierung)} \\
& e_{min} \leq e \leq e_{max} && \text{Exponentenbereich}
\end{align*}

Schreibweise:
\begin{equation*}
x = \sigma\,(0.d_1d_2d_3\,\cdots\,d_t)b^e
\end{equation*}

\para{Denormalisierte GPZen}
\missingfigure{zahlenstrahl}

\begin{align*}
x_d = \sigma\,M\,b^{e_{min}} &\\
M = \sum^t_{k=2} d_k b^{-k} \hspace{0.5cm} \text{mit} \hspace{0.5cm} & d_2,\cdots,d_t\, \in\,\{0,1,\cdots,b-1\} \\
& \text{keine Normierung} 
\end{align*}

\para{Gleitpunktzahlsystem}
\Fhatfunc{b,t,$e_{min}$,$e_{max}$}

\example{Beispiel: "`Doppeltes Grundformat"' ("`double"' in C, Java, MATLAB)}

\Fhatfunc{2,53,-1021,1024} Im PC als 64bit $\rightarrow$ 52 + 1 da erstes bit ungleich 0 sein muss, 11 Stellen für Exponenten / Umschaltung zu denorm. GPZ. Symbolische Ausdrücke ("`Inf"',"'NaN"',...) 
kleinste/größte pos Zahl:
\begin{align*}
	& x_{min}(\F) =  2^{-1022}\,\approx\, 2,23\cdot\,10^{-308} \\
	& x_{max}(\F) \approx 1,8\,\cdot\,10^{308} \\
  & x_{min}(\Fhat) \approx 5\cdot\,10^{-324} & \text{nicht in diesen Bereich gehen!}
\end{align*}

\subsubsection{Rundung}
Die Abbildung $rd: \mathbb{R} \rightarrow \Fhat, x\,\mapsto\,rd(x)$ heißt Rundung, wenn $|rd(x)-x|=min|y-x|, y\in\,\F$. 
\todo{realtiver Fehler unterbringen}
\begin{empheq}[innerbox=\fbox,right=\Leftarrow{\text{gilt nur für normalisierte Zahlen}}]{align*}
\text{Für} \hspace{1cm} x_{min}(\F)&\leq\,|x|\,\leq\,x_{max}\,(\F) \hspace{1cm}\text{gilt:}\\
\frac{|rd(x)-x|}{x}&\leq \frac{1}{2}\,b^{-t+1} \eqqcolon eps \leftarrow \text{Maschinengenauigkeit}
\end{empheq}

Es sei $\frac{rd(x)-x}{x}=\sigma\,\Rightarrow\,rd(x)=x(1+\delta)$
es gilt:
\begin{empheq}[innerbox=\fbox]{align*}
rd(x)=x(1+\delta) & \hspace{1cm} |\delta|\leq\,eps
\end{empheq}

\example{Maschinengenauigkeit für "`double"'}
\Fhatfunc{2,53,-1021,1024}
\begin{equation*}
eps = \frac{1}{2}\,2^{-53+1}\approx\,1,1\cdot\,10^{-16}
\end{equation*}
Maschinenepsilon 
\begin{equation*}
E_M=2 eps = b^{-t+1}
\end{equation*}

$E_M$ ist der Abstand zwischen 1 und der nächstgrößeren GPZ

\subsubsection{Gleitpunktarithmetik}
