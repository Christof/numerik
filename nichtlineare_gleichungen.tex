\section{Numerik nichtlinearer Gleichungen und Gleichungssysteme}
\subsection{Grundlagen}
Ziel: Gesucht $x^* \in M \subset \rn$, so dass $f(x^*) = 0$ wobei $f: M \rightarrow \rn$\\
($ h(x) = g(x) \Leftrightarrow f(x) := h(x) - g(x) = 0$)\\
Typische ``numerische'' Vorgehensweise:
\begin{align*}
  f(x) = 0 \overset{G(x) \neq 0}{\Leftrightarrow} -G(x)f(x) = 0 \Leftrightarrow \underbrace{x - G(x)f(x)}_{=: \Phi(x) \text{ Fixpktglg}} = x
\end{align*}
Fixpunktiteration: $x^{(k+1)} = \Phi(x^{(k)})$\\
Fragestellungen:
\begin{itemize}
  \item Konvergenz
  \item Konvergenzbereich (für welche $x^{(0)}$ konvergiert die FPI?)
  \item Konvergenzgeschwindigkeit
\end{itemize}
\definition: Sei $M \subset \rn,\;\Phi: M \rightarrow \rn,\;\Phi(x^*) = x^* \in M$ und $x^{(k+1)} := \Phi(x^{(k)})$\\
Die Folge $(x^{(k)})_{k \in \mathbb{N}_0}$ heißt
\begin{itemize}
  \item global konvergent, falls $\mathbin{\textcolor{rot}{\forall\;x^{(0)} \in M}}: \underset{k \rightarrow \infty}{\lim} x^{(k)} = x^*$
  \item lokal konvergent, falls $\exists\;\epsilon > 0 \quad \mathbin{\textcolor{rot}{\forall\;x^{(0)} \in U_\epsilon(x^*)}}: \underset{k \rightarrow \infty}{\mathrm{lim}} x^{(k)} = x^*$
  \item linear konvergent (Konvergenzordnung $p=1$), falls $\exists\; q \in [0,1)\quad \norm{x^{(k+1)} - x^*} \leq q \norm{x^{(k)} - x^*} \quad \forall\; k \geq k_0$
      ($q$ Konvergenzfaktor)
  \item von Konvergenzordnung $p > 1$, falls $\underset{k \rightarrow \infty}{\lim} x^{(k)} = x^*$ und $\exists\;c > 0$ mit
    $\norm{x^{(k+1)} - x^*} \leq c \norm{x^{(k)} - x^*}^p$ ($p=2$: quadratische Konvergenz)
  \item Fixpunktiterationen sind i.A. nur lokal konvergent, garantierter Konvergenzbereich $D$ lässt
    sich theoretisch über Banach'schen Fixpunktsatz angeben: $\Phi_D: D \rightarrow D$ und $\Phi$ ist auf $D$ eine Kontraktion,
    d.h. $\exists\; q \in [0,1): \norm{\Phi(x) - \Phi(y)} \leq q\norm{x - y} \quad \forall\;x,y \in D$
    In der Praxis kennt man $D$ nicht.
  \item Fixpunktiterationen sind mindestens linear konvergent (wenn sie konvergieren)\\
    $\norm{x^{(k+1)} - x^*} = \norm{\Phi(x^{(k)} - \Phi(x^*)} \leq q \norm{x^{(k)} - x^*}$
\end{itemize}
Abbruchkriterien bei Fixpunktiteration:
\begin{enumerate}[a)]
  \item ``Residuum'' $\norm{f(x^{(k)}} \leq tol$
  \item bei quadratischer Konvergenz gilt: $\frac{\norm{x^{(k+1)} - x^{(k)}}}{\norm{x^k - x^*}} \underset{k \rightarrow \infty}{\longrightarrow} 1$
    Zähler vom Bruch ist Abbruchkriterium
\end{enumerate}
\missingfigure{Abbruchkriterium Fixpunktiteration}

